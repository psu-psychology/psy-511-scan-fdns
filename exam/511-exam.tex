\documentclass[]{exam}
\usepackage{graphicx}
\usepackage{wrapfig}
\usepackage[utf8]{inputenc}

\title{PSYCH 511 Exam}
\author{}
\date{November 9, 2015}

\pagestyle{headandfoot}
\firstpageheader{PSY 511}{}{Exam}
\runningheader{PSY 511}{}{Exam}
\firstpagefooter{}{Page \thepage}{}
\runningfooter{}{Page \thepage}{}

\begin{document}
\maketitle

\begin{center}
  \fbox{\fbox{\parbox{5.5in}{\centering
        You may take up to 4 hours to complete this exam. You \emph{may} use your book, notes, or other resources, but \emph{not} another person. You may either print the exam and turn it in to me personally, or deposit it in my mailbox in Moore, or, you may send it to me electronically. Exams are due by midnight, November 16, 2015.}}}
\end{center}
\vspace{0.1in}
\makebox[\textwidth]{Name:\enspace\hrulefill}

\newpage

\begin{questions}

\section{Multiple Choice}

\question The myelin sheath is formed by \fillin in the central nervous system and by \fillin in the peripheral nervous system.
\begin{choices}
\choice  Oligodendrocytes; Schwann cells.
\choice  Schwann cell; Oligodendrocytes.
\choice  Astrocytes; Schwann Cells.
\choice  Stellate cells; Microglial cells.
\end{choices}

\question  At the \emph{peak} of the action potential... 
\begin{choices}
\choice Both the forces of diffusion and electrostatic pressure tend to move K+ out of the cell.
\choice The force of diffusion tends to move Cl- outside the cell.
\choice Both the forces of diffusion and electrostatic pressure tend to move Na+ outward.
\choice The force of diffusion tends to move K+ in; the force of electrostatic pressure moves K+ out.
\end{choices}

\question  When a neuron is at rest, Cl- concentration is highest outside the cell.  Thus, when Cl- channels open, Cl- ions move \fillin the cell, creating an \fillin effect on the cell.
\begin{choices}
\choice  into; inhibitory
\choice  into; excitatory
\choice  out of; inhibitory
\choice  out of; inhibitory
\end{choices}

\question  If you wanted to know how the brain responds on a millisecond-to-millisecond basis to a stimulus but didn’t care as much about where the activity was occurring, you should choose \fillin.  If you needed to know what brain regions were active in response to a stimulus, but cared less about knowing the specific time course, you should choose \fillin. 
\begin{choices}
\choice  EEG; Functional MRI (fMRI).
\choice  fMRI; PET 
\choice  Structural MRI; MEG
\choice  NIRS; ERP
\end{choices}

\question  A toxin in pufferfish, a Japanese delicacy, blocks voltage-gated Na+ channels.  Why might this be bad for you?
\begin{choices}
\choice  It could turn IPSPs into EPSPs.
\choice  It would slow neurotransmitter movement across the synaptic cleft.
\choice  It would block action potentials.
\choice  It would prevent the undershoot phase of the action potential.
\end{choices}

\question  Which of the following statements about synaptic communication is CORRECT?
\begin{choices}
\choice  Ionotropic receptors contain only a neurotransmitter binding site, not an ion channel.
\choice  Most neurons contain gap junctions or direct electrical coupling sites with other neurons. 
\choice  Metabotropic receptors are faster-acting than ionotropic receptors.
\choice  Metabotropic receptors activate ion channels separate from the receptor itself though chemical messengers called G-proteins.
\end{choices}

\newpage

\question  Humans have a high encephalization factor.  What does this mean?
\begin{choices}
\choice Our brains are bigger than our spinal cords.
\choice Our heads at birth are bigger than they should be for safe passage down the birth canal.
\choice Our skulls are too small to contain our cerebral cortex.
\choice Our brains are bigger than other mammals of similar body size.
\end{choices}

\question  Depolarization of the presynaptic terminal leads to an influx of \fillin, which causes exocytosis or the fusion of  \fillin with the presynaptic membrane.
\begin{choices}
\choice  Na+; Ligands. 
\choice  Ca2+; Synaptic vesicles.
\choice Ca2+; Neurotransmitters.
\choice  Na+; G-proteins.
\end{choices}

\question  The fiber bundle that connects the left and right hemisphere is called the \fillin, while the deep groove that divides frontal and temporal lobes is called the \fillin. 
\begin{choices}
\choice  corpus callosum; Sylvian (lateral) fissure
\choice  the Sylvian (lateral) fissure; the central sulcus
\choice  corpus callosum; the central sulcus
\choice  longitudinal fissure; corpus callosum
\end{choices}

\vspace{.25in}

Please put in their proper order the steps that lead to synaptic communication between neurons using the choices below.

\begin{choices}
\choice Voltage-gated Ca++ channels open
\choice Action potential propagates down presynaptic axon.
\choice Ca++ initiates exocytosis of neurotransmitter
\choice Ligand-gated receptors bind neurotransmitter and   activate channels in the postsynaptic cell
\choice Neurotransmitter defuses across synaptic cleft
\end{choices}

\question  Step 1

\question  Step 2

\question  Step 3

\question  Step 4

\question  Step 5	

\newpage
\question  Parkinson’s and Schizophrenia are both linked to a disturbance in one of the \fillin pathways projecting from the \fillin to the forebrain.   
\begin{choices}
\choice  dopamine; midbrain tegmentum
\choice  acetylcholine; midbrain tectum
\choice  serotonin; mesolimbocortex
\choice  dopamine; locus coeruleus
\end{choices}

\question  When an unknown drug is applied to the synapse between a nerve and skeletal muscle, activity in the motor neuron stops.  The drug is an \fillin and could operate by \fillin.
\begin{choices}
\choice  agonist; blocking neurotransmitter release
\choice  antagonist; blocking reuptake mechanisms
\choice  agonist; blocking enzyme degeneration of transmitter
\choice  antagonist; blocking postsynaptic receptors
\end{choices}

\question  Segmental organizational schemes are characteristic of the nervous systems of which animal groups?
\begin{choices}
\choice  vertebrates alone.
\choice  selected invertebrates -- worms, arthropods.
\choice  vertebrates and selected invertebrates.
\choice  no animal group has a segmental organization to its nervous system.
\end{choices}

\question  fMRI \fillin measures the activity of \fillin neurons in a particular region of the brain by means of a magnetic signal associated with differences in \fillin oxygen levels.
\begin{choices}
\choice  indirectly; individual; CSF.
\choice  directly; individual; blood.
\choice  indirectly; large groups; blood
\choice  directly; large groups; BOLD.
\end{choices}

\newpage
\section{Identification}

\question Identify the structure in the images below.

\begin{figure}[h]
\includegraphics[width=0.80\textwidth]{exam-1-fig-1.jpg}
\centering
\end{figure}

\question Identify the structure in the images below.

\begin{figure}[h]
\includegraphics[width=0.80\textwidth]{exam-1-fig-2.jpg}
\centering
\end{figure}

\question Identify the structure in the images below.

\begin{figure}[h]
\includegraphics[width=0.80\textwidth]{exam-1-fig-3.jpg}
\centering
\end{figure}

\newpage
\section{Short answer}

Please answer \emph{both of} the following questions in a few short paragraphs.  You may create a small figure or two if it helps you.

\question  Describe the events that lead to and occur during an action potential.
\vspace{2.5in}
\question  Describe the principal components of the cerebral cortex and the external landmarks (fissures and sulci) that separate one part from the other.
\vspace{2.5in}

\newpage
\section{Short answer}

Please answer one (1) of the following questions in a few short paragraphs.

\question  What are the main types of glial cells, where in the nervous system are they found, and what are their main functions?
\vspace{2.5in}
\question  Describe the main components of the limbic system and their anatomical locations.  Why are these components thought to be functionally related to one another?
\vspace{2.5in}
\question  Describe the main components of the midbrain, their locations relative to one another and an important functional role associated with each major structure.
\vspace{2.5in}

\newpage
\section{Short answer}

Please answer one (1) of the following questions in a few short paragraphs.

\question  How does the action potential propagate down the axon in an unmyelinated neuron?  How does the process differ in a myelinated axon?
\vspace{2.5in}
\question  What sorts of messages are received by the receiving or postsynaptic neuron?  How are these messages generated?  Describe an example of the postsynaptic effect a specific neurotransmitter might have.
\vspace{2.5in}
\question  Describe the two types of refractory periods neurons show.  How does the fact that the action potential has an absolute refractory period and a stereotyped amplitude influence the kind of message a neuron can convey? 

\newpage
\section{Short answer}

Please answer one (1) of the following questions in a few short paragraphs.

\question  Which three neurotransmitters are released at the majority of synapses in the nervous system?  What are their basic functions?
\vspace{2.5in}
\question  What distinguishes a neuromodulator from other types of neurotransmitters?  Give two examples of neuromodulators, brief descriptions of some of their known functions, and where in the brain these cells are clustered.
\vspace{2.5in}
\question  Explain how metabotropic receptors differ from ionotropic ones.  Give an example of a neurotransmitter system that depends on both types.

\newpage
\section{Short answer}

Please answer one (1) of the following questions in a few short paragraphs.

\question Summarize the key behavioral and neural features of depression, bipolar disorder, \emph{or} schizophrenia. Briefly compare the disorder you describe to one of the other disorders.
\vspace{2.5in}
\question Summarize the key behavioral and neural features of Parkinson's disease \emph{or} Huntington's disease. Briefly compare the disorder you describe to one of the other disorders.
\vspace{2.5in}

\newpage
\section{Short answer}

Please answer one (1) of the following questions in a few short paragraphs.

\question What are the primary goals of the nervous system's input or sensory systems? Describe at least two ways that different sensory systems are similar to one another.
\vspace{2.5in}
\question Are cognition and emotion different facets of the same type of processing? Briefly argue for or against the proposition.

\newpage
\section{Bonus}

Answer one (1) additional short answer question from Sections 3-7.

\end{questions}
\end{document}